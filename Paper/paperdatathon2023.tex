\documentclass[a4paper]{article}

\usepackage[dvipsnames]{xcolor}
\usepackage[framemethod=TikZ]{mdframed}

\usepackage[utf8]{inputenc}
\usepackage[T1]{fontenc}
\usepackage{textcomp}
\usepackage{amsmath, amssymb}
\usepackage{mdframed}
\usepackage{pgfplots}
\usepackage{import}
\usepackage{mathrsfs}
\usepackage{xifthen}
\usepackage{float}
\usepackage{nicematrix}
\usepackage{pdfpages}
\usepackage{tcolorbox}
\usepackage{transparent}
\pgfplotsset{compat=1.18}
\usepackage{tikz}
\usepackage[left=1cm, right=1cm, top=2cm, bottom=2cm]{geometry}

\newcommand\Warning{%
 \makebox[1em][c]{%
 \makebox[0pt][c]{\raisebox{.1em}{\small!}}%
 \makebox[0pt][c]{\color{red}\Large$\bigtriangleup$}}}%



%Theorem
\usepackage{amsthm}

\makeatother
\usepackage{thmtools}
\usepackage[framemethod=TikZ]{mdframed}
\mdfsetup{skipabove=1em,skipbelow=0em}


\theoremstyle{definition}


\declaretheoremstyle[
    headfont=\bfseries\sffamily\color{BrickRed!70!black}, bodyfont=\normalfont,
    mdframed={
        linewidth=2pt,
        rightline=false, topline=false, bottomline=false,
        linecolor=BrickRed, backgroundcolor=BrickRed!20,
    }
]{thmreallyredbox}

\declaretheoremstyle[
    headfont=\bfseries\sffamily\color{ForestGreen!70!black}, bodyfont=\normalfont,
    mdframed={
        linewidth=2pt,
        rightline=false, topline=false, bottomline=false,
        linecolor=ForestGreen, backgroundcolor=ForestGreen!3,
    }
]{thmgreenbox}


\declaretheoremstyle[
    headfont=\bfseries\sffamily\color{Fuchsia!70!black}, bodyfont=\normalfont,
    mdframed={
        linewidth=2pt,
        rightline=false, topline=false, bottomline=false,
        linecolor=Fuchsia, backgroundcolor=Fuchsia!5,
    }
]{thmfuchsiabox}

\declaretheoremstyle[
    headfont=\bfseries\sffamily\color{NavyBlue!70!black}, bodyfont=\normalfont,
    mdframed={
        linewidth=2pt,
        rightline=false, topline=false, bottomline=false,
        linecolor=NavyBlue, backgroundcolor=NavyBlue!5,
    }
]{thmbluebox}

\declaretheoremstyle[
    headfont=\bfseries\sffamily\color{NavyBlue!70!black}, bodyfont=\normalfont,
    mdframed={
        linewidth=2pt,
        rightline=false, topline=false, bottomline=false,
        linecolor=NavyBlue
    }
]{thmblueline}

\declaretheoremstyle[
    headfont=\bfseries\sffamily\color{RawSienna!70!black}, bodyfont=\normalfont,
    mdframed={
        linewidth=2pt,
        rightline=false, topline=false, bottomline=false,
        linecolor=RawSienna, backgroundcolor=RawSienna!5,
    }
]{thmredbox}

\declaretheoremstyle[
    headfont=\bfseries\sffamily\color{RawSienna!70!black}, bodyfont=\normalfont,
    numbered=no,
    mdframed={
        linewidth=2pt,
        rightline=false, topline=false, bottomline=false,
        linecolor=RawSienna, backgroundcolor=RawSienna!2,
} 
]{thmproofbox}

\declaretheoremstyle[
    headfont=\bfseries\sffamily\color{NavyBlue!70!black}, bodyfont=\normalfont,
    numbered=no,
    mdframed={
        linewidth=2pt,
        rightline=false, topline=false, bottomline=false,
        linecolor=NavyBlue, backgroundcolor=NavyBlue!2,
    },
]{thmexplanationbox}


\declaretheorem[style=thmfuchsiabox, name=Definition]{definition}
\declaretheorem[style=thmbluebox, numbered=no, name=Example]{eg}
\declaretheorem[style=thmredbox, name=Proposition]{prop}
\declaretheorem[style=thmredbox, name=Theorem]{theorem}
\declaretheorem[style=thmredbox, name=Lemma]{lemma}
\declaretheorem[style=thmredbox, numbered=no, name=Corollary]{corollary}

\declaretheorem[style=thmproofbox, name=Proof]{replacementproof}
\renewenvironment{proof}[1][\proofname]{\vspace{-10pt}\begin{replacementproof}}{\end{replacementproof}}


\declaretheorem[style=thmexplanationbox, name=Explanation]{tmpexplanation}
\newenvironment{explanation}[1][]{\vspace{-10pt}\begin{tmpexplanation}}{\end{tmpexplanation}}

\declaretheorem[style=thmreallyredbox, numbered=no, name=\Warning$\text{ }$Remark \Warning]{remark}
\declaretheorem[style=thmgreenbox, numbered=no, name=Note]{note}

\newtheorem*{uovt}{UOVT}
\newtheorem*{notation}{Notation}
\newtheorem*{previouslyseen}{As previously seen}
\newtheorem*{problem}{Problem}
\newtheorem*{observe}{Observe}
\newtheorem*{property}{Property}
\newtheorem*{intuition}{Intuition}


\usepackage{etoolbox}
\AtEndEnvironment{vb}{\null\hfill$\diamond$}%
\AtEndEnvironment{intermezzo}{\null\hfill$\diamond$}%
% \AtEndEnvironment{opmerking}{\null\hfill$\diamond$}%

% http://tex.stackexchange.com/questions/22119/how-can-i-change-the-spacing-before-theorems-with-amsthm
\makeatletter
% \def\thm@space@setup{%
%   \thm@preskip=\parskip \thm@postskip=0pt
% }

\newcommand{\oefening}[1]{%
    \def\@oefening{#1}%
    \subsection*{Oefening #1}
}

\newcommand{\suboefening}[1]{%
    \subsubsection*{Oefening \@oefening.#1}
}

\newcommand{\exercise}[1]{%
    \def\@exercise{#1}%
    \subsection*{Exercise #1}
}

\newcommand{\subexercise}[1]{%
    \subsubsection*{Exercise \@exercise.#1}
}


\usepackage{xifthen}

\def\testdateparts#1{\dateparts#1\relax}
\def\dateparts#1 #2 #3 #4 #5\relax{
    \marginpar{\small\textsf{\mbox{#1 #2 #3 #5}}}
}

\def\@lesson{}%
\newcommand{\lesson}[3]{
    \ifthenelse{\isempty{#3}}{%
        \def\@lesson{Lecture #1}%
    }{%
        \def\@lesson{Lecture #1: #3}%
    }%
    \subsection*{\@lesson}
    \testdateparts{#2}
}

% \renewcommand\date[1]{\marginpar{#1}}


\makeatother






% figure support
\usepackage{import}
\usepackage{xifthen}
\usepackage{collectbox}
\usepackage{enumerate}
\usepackage{amsfonts}
\pdfminorversion=7
\usepackage{fancyhdr}



\newcommand{\incfig}[2][1]{%
    \def\svgwidth{#1\columnwidth}
    \import{./figures/}{#2.pdf_tex}
}

\pdfsuppresswarningpagegroup=1

\pagestyle{fancy}
\fancyhf{}
\rhead{NTT data}
\lhead{Datathon 2023}
\rfoot{Page \thepage}

\makeatletter
\renewcommand*\env@matrix[1][*\c@MaxMatrixCols c]{%
  \hskip -\arraycolsep
  \let\@ifnextchar\new@ifnextchar
  \array{#1}}
\makeatother

\def\undertilde#1{\mathord{\vtop{\ialign{##\crcr
$\hfil\displaystyle{#1}\hfil$\crcr\noalign{\kern1.5pt\nointerlineskip}
$\hfil\tilde{}\hfil$\crcr\noalign{\kern1.5pt}}}}}


\newcommand{\mybox}{%
    \collectbox{%
        \setlength{\fboxsep}{1pt}%
        \fbox{\BOXCONTENT}%
    }%
}

\setlength\parindent{0pt}

\pdfsuppresswarningpagegroup=1

\title{Datathon 2023}
\author{NTT data}
\date{November 2023}

\begin{document}
\definecolor{xdxdff}{rgb}{0.49019607843137253,0.49019607843137253,1.}
\definecolor{ududff}{rgb}{0.30196078431372547,0.30196078431372547,1.}
\maketitle

\newpage
\tableofcontents


\newpage
\section{Supositions}
\begin{itemize}
	\item Price of product in year $n$ is last year's price multiplied by a ratio  $\gamma$ (inflation)
	\item Central storage center that supplies all hospitals (stores their "almacenable" products)
	\item Environtmental cost \quad $\propto$ \quad $\#$orders
	\item Unlimited transit (limited indirectly by  maximum capacity of storage center) 
	\item Products are distributed to hospitals in a uniform way (\underline{Not exactly: elaborate later})
	\item Mensual orders
\end{itemize}

\section{Extras}
\begin{itemize}
	\item For products of "transito" we group them by day manually
\end{itemize}



\newpage
\section{Parameters}

\begin{note}
Vamos a agrupar las compras de todos los hospitales en una única compra centralizada
\end{note}

\vspace{5mm}

Initial definition:
\begin{itemize}
	\item $I = \left\{ \text{index of product} \right\} $ 
\end{itemize}

\vspace{5mm}

Constantes:
\begin{itemize}
	\item $\gamma$: factor multiplicativo del precio de año a año (inflación + \ldots)
	\item $c_i$: storage cost for a product  $i$ (unitary) for a day
	\item $C_{max}^{i}$: maximum quantity of product $i$ that we can store
\end{itemize}


\vspace{5mm}

Precalculados
\begin{itemize}
\item  $v^{i}\left( t \right) $: consumption velocity of product $i$ at time  $t$
\item $\xi^{i}\left( t \right) $: all hospitals unified demand of product $i$ at time  $t$
\end{itemize}

\vspace{5mm}


Constants to change by client:
\begin{itemize}
	\item $\beta$: extra quantity factor in order (resilience)
	\item  $P_{max}$ : $\#$ orders \quad  $\propto$ \quad $CO_2$ emissions
\end{itemize}


\newpage
\section{Variables}
\begin{itemize}
	\item $p^{i}\left( t \right) $ : quantity of product $i$ demanded at time  $t \in \left\{ 1,\ldots,12 \right\} $
	\item  $\delta\left( t \right) $: boolean (binary) variable to determine if there is an order at time $t \in \left\{ 1,\ldots,12 \right\} $
\end{itemize}

\vspace{15mm}

\section{Restrictions}

\begin{remark}
Fijamos $i$. Ajustamos un modelo para cada producto
\end{remark}


\begin{definition}
We define $S^{i}\left( t \right) $ as the inventory of product  $i$ at time  $t$ (En el dia 2 del mes):
 \begin{align*}
	&S^{i}\left( t \right) = - \sum_{t'=1}^{t-1} v^{i}\left( t' \right)  + \sum_{t'=1}^{t} \delta\left( t' \right) \cdot p^{i}\left( t' \right) 
.\end{align*}
\end{definition}


\vspace{5mm}

Capacity restriction:
\begin{align*}
	&S^{i}\left( t \right) \le C^{i}_{max}
.\end{align*}

Restriction to verify we have enough inventory
\begin{align*}
	&S^{i}\left( t \right) + \sum_{t'=1}^{t-1} v^{i}\left( t' \right) \ge \beta\cdot \sum_{t'=1}^{t} \xi^{i}\left( t' \right) 
.\end{align*}


Restriction to only do $P_{max}$ orders:
\begin{align*}
	&\sum_{t=1}^{12} \delta\left( t \right) = P_{max}
.\end{align*}

Basic restricitions:
\begin{align*}
	&S^{i}\left( t \right) \ge  0\\
	&p^{i}\left( t \right) \ge 0
.\end{align*}

Restrictions over variables:
\begin{align*}
	&\delta \left( t \right) \in  \left\{ 0,1 \right\} 
.\end{align*}


\begin{remark}
ESTO SE DEBE VERIFICAR $\forall t = 1,\ldots,12$
\end{remark}

\vspace{15mm}

\section{Objetivos}

\begin{itemize}
	\item \underline{Minimizar precio}
\end{itemize}

Formula a minimizar:
\begin{align*}
	&\min \sum_{t}^{} S^{i}\left( t \right) \cdot c_i
.\end{align*}

\end{document}



